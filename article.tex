% This is a simple sample document.  For more complicated documents take a look in the exercise tab. Note that everything that comes after a % symbol is treated as comment and ignored when the code is compiled.

\documentclass{article} % \documentclass{} is the first command in any LaTeX code.  It is used to define what kind of document you are creating such as an article or a book, and begins the document preamble

\usepackage{amsmath} % \usepackage is a command that allows you to add functionality to your LaTeX code

\IfFileExists{latexml.sty}{
  \usepackage{latexml}
}{
\newcommand{\lxAddClass}[1]{}
}


\title{\textit{STOCHASTISCHE} RECHENÜBUNGEN} % Sets article title
%\author{My Name} % Sets authors name
%\date{\today} % Sets date for date compiled

% The preamble ends with the command \begin{document}
\begin{document} % All begin commands must be paired with an end command somewhere
    
    \maketitle % creates title using information in preamble (title, author, date)
    \textit{Eine Sammlung von Aufgaben zu grundlegenden Konzepten der Stochastik.}
    
    \section*{Endliche Wahrscheinlichkeitsräume} % creates a section
    
    
    Hier könnte ein kleiner Wiederholungstext mit Definitionen und den wichtigsten Sätzen stehen....
    Today I am learning \LaTeX.
    \LaTeX{} is a great program for writing math. I can write in line math such as $a^2+b^2=c^2$ %$ tells LaTexX to compile as math
     . I can also give equations their own space: 
    \begin{equation*} % Creates an equation environment and is compiled as math
    \gamma^2+\theta^2=\omega^2=\sum_{i=1}^N \sum_{j=2}^K G^{M,N} (f(x)-f(y)) = -\frac{1}{10}k
    \end{equation*}
    If I do not leave any blank lines \LaTeX{} will continue  this text without making it into a new paragraph.  Notice how there was no indentation in the text after equation (1).  
    Also notice how even though I hit enter after that sentence and here $\downarrow$
    \LaTeX{} formats the sentence without any break.  Also   look  how      it   doesn't     matter          how    many  spaces     I put     between       my    words.
    
    \subsection*{Aufgabe: Urnen}
    Betrachten Sie eine Urne mit 8 Kugeln, nummeriert von 1 bis 8.
    Ziehe zufällig eine Kugel. Sei $\Omega$ die Menge der möglichen Ergebnisse.
    Definiere die Ereignisse
    $$A = \{\text{Ergebnis ist mindestens 5}\},$$
    $$B = \{\text{Ergebnis ist höchstens 7}\},$$
    $$C = \{\text{Ergebnis ist ungerade}\}.$$
    Geben Sie die folgenden Ereignisse durch Auflisten ihrer Elemente an: a) $\Omega$, b) $A \cap B$, c) $B \cup C$.

    \subsubsection*{a)}
    Lösung zur ersten Teilaufgabe hier.
    \begin{equation*}
     A = ...
    \end{equation*}


    \subsubsection*{b)}
    Using equation ... we get that la la la 

    \subsubsection*{c)}
    Noch eine weitere Lösung einer Teilaufgabe.

    
    \subsection*{Aufgabe: technische Defekte}
    Ein technisches Gerät besteht aus n verschiedenen Komponenten, nummeriert mit $i =
    1, . . . , n$. Es bezeichne $D_i$ das Ereignis, dass die $i$-te Komponente defekt ist. Stellen Sie
    die folgenden Ereignisse mittels elementarer Mengenoperationen dar: a) mindestens eine
    Komponente defekt, b) genau eine Komponente defekt, c) höchstens eine Komponente
    defekt, d) höchstens eine Komponente nicht defekt.

    \subsubsection*{a)}
    \begin{equation*}
     A = ...
    \end{equation*}

    \subsubsection*{b)}
    \begin{equation*}
     B = ...
    \end{equation*}

    \subsubsection*{c)}
    \begin{equation*}
     C = ...
    \end{equation*}

    \subsubsection*{d)}
    \begin{equation*}
     D = ...
    \end{equation*}


    \section*{Bedingte Wahrscheinlichkeit und Unabhängigkeit} % creates a section
    
    \textbf{Hello World!} Today I am learning \LaTeX. %notice how the command will end at the first non-alphabet charecter such as the . after \LaTeX
     \LaTeX{} is a great program for writing math. I can write in line math such as $a^2+b^2=c^2$ %$ tells LaTexX to compile as math
     . I can also give equations their own space: 
    \begin{equation*} % Creates an equation environment and is compiled as math
    \gamma^2+\theta^2=\omega^2=\sum_{i=1}^N \sum_{j=2}^K G^{M,N} (f(x)-f(y)) = -\frac{1}{10}k
    \end{equation*}
    If I do not leave any blank lines \LaTeX{} will continue  this text without making it into a new paragraph.  Notice how there was no indentation in the text after equation (1).  
    Also notice how even though I hit enter after that sentence and here $\downarrow$
     \LaTeX{} formats the sentence without any break.  Also   look  how      it   doesn't     matter          how    many  spaces     I put     between       my    words.
    
    \paragraph{Test Paragraph}\lxAddClass{sometstclass}
    This will be a test paragraph.

    \section*{Diskrete Zufallsvariablen und Verteilungen}

    \section*{Gemeinsame Verteilung, Unabhängigkeit von Zufallsvariablen}

    \section*{Kenngrößen für Zufallsvariablen}

    \section*{Zufallsvariablen mit Dichte}

    \section*{Grenzwertsätze}


\end{document} % This is the end of the document

